\documentclass[10pt,pdf,hyperref={unicode},aspectratio=169]{beamer}

\usepackage [T2A] {fontenc}
\usepackage [utf8] {inputenc}
\usepackage [english,russian] {babel}
\usepackage {textcase}
\usepackage {listings} 
\usepackage {alltt}
%\usetheme   {Berlin}
\usetheme   {Warsaw}
 
\begin{document}

\title{Реализация FRP библиотеки на языке Erlang}  

\institute{Институт математики, механики и компьютерных наук им. И.И. Воровича \\
    \vspace{0.3cm}
    Кафедра информатики и вычислительного эксперимента  \\
    Научный руководитель:  Брагилевский В. Н.\\
    \vspace{0.3cm}
}

\author{Нежевский Н.М.}

\date{9 апреля 2015}

\begin{frame}
  \titlepage
\end{frame}

\begin{frame}
\frametitle{Введение}
\framesubtitle{Функциональное реактивное программирование}
\begin{itemize}
\item Потоки данных
\item Событийность
\item Преобразование данных
\end{itemize}
\end{frame}

\begin{frame}[fragile]
\frametitle{Введение}
\framesubtitle{Функциональное реактивное программирование}
\onslide<1->

\begin{columns}
\column{0.45\textwidth}
 \begin{block}{Обычный подход}
 \begin{alltt}
  a = 1; b = a + 1;
  \textcolor{cyan}{// a = 1, b = 2}
  a = 2;
  \textcolor{cyan}{// a = 2, b = 2}
 \end{alltt}
 \end{block}

\column{0.45\textwidth}
 \begin{block}{Реактивный подход}
 \begin{alltt}
  a = 1; b = a + 1;
  \textcolor{cyan}{// a = 1, b = 2}
  a = 2;
  \textcolor{cyan}{// a = 2, b = 3}
 \end{alltt}
 \end{block}
\end{columns}

\onslide<2->
 \begin{block}{Функциональный реактивный подход}
  \begin{center}
   g(f(z(x)))
  \end{center}
 \end{block}
\end{frame}

\begin{frame}
 \frametitle{Принципы реализации}
 \framesubtitle{Общие принципы реализации}
 \begin{itemize}
  \item Структура
  
  \item События
  
  \item Логика  

  \item Распространение изменений
  
 \end{itemize}
\end{frame}

\begin{frame}[fragile]

 \frametitle{Пример: обработка данных}
 \framesubtitle{Узлы}
\begin{block}{Создание и описание узлов}
\fontsize{8pt}{8pt}\selectfont
\begin{alltt}

\textcolor{cyan}{%% node() :: pid() | fun()}
 \{ok, \textcolor{magenta}{Entry}\} = frp_api:start_node(),

 \{ok, \textcolor{magenta}{RecG}\}  = frp_api:start_node(\textcolor{blue}{fun}(\textcolor{magenta}{State}, \textcolor{magenta}{_Value}) -> 
                                      main:receive_new_data(\textcolor{magenta}{State}, \textcolor{magenta}{_Value}) 
                                  \textcolor{blue}{end}, \{{\textcolor{green}{"NASDAQ"}, \textcolor{green}{"GOOG"}}\}),
                                  
 \{ok, \textcolor{magenta}{HandG}\} = frp_api:start_node(\textcolor{blue}{fun}(\textcolor{magenta}{State}, \textcolor{magenta}{_Value}) -> 
                                      main:handle_new_data(\textcolor{magenta}{State}, \textcolor{magenta}{Value}) 
                                  \textcolor{blue}{end}),                                  
                                  
 \{ok, \textcolor{magenta}{SendDB}\} = frp_api:start_node(\textcolor{blue}{fun}(\textcolor{magenta}{State}, \textcolor{magenta}{_Value}) -> 
                                      receiver:send_to_db(\textcolor{magenta}{State}, \textcolor{magenta}{Value}) 
                                   \textcolor{blue}{end}),                                                                    
\end{alltt}

\end{block} 

\end{frame}

\begin{frame}[fragile]

 \frametitle{Пример: обработка данных}
 \framesubtitle{Сеть}
\begin{block}{Создание сети}
\fontsize{8pt}{8pt}\selectfont
\begin{alltt}
\textcolor{cyan}{%% network() :: digraph()}
\textcolor{cyan}{%% vertex()  :: unique_id() | node()}

\textcolor{magenta}{Network} = frp_api:create_network(),
\textcolor{magenta}{Nodes} = [\{entry, \textcolor{magenta}{Entry}\}, \{receiver, \textcolor{magenta}{RecG}\}, \{handler, \textcolor{magenta}{HandG}\}, \{dbsender, \textcolor{magenta}{SendDB}\}],
frp_api:add_nodes(\textcolor{magenta}{Network}, \textcolor{magenta}{Nodes}),
\end{alltt}

\end{block} 

\begin{block}{Установление зависимостей}
\fontsize{7pt}{7pt}\selectfont
\begin{alltt}
frp_api:add_listener(\textcolor{magenta}{Network}, entry, receiver),
frp_api:add_listener(\textcolor{magenta}{Network}, receiver, handler),
frp_api:add_listener(\textcolor{magenta}{Network}, handler, dbsender),
\end{alltt}

\end{block} 

\end{frame}

\begin{frame}[fragile]
 \frametitle{Пример: обработка данных}
 \framesubtitle{События}

\begin{block}{Однократное событие}
\fontsize{8pt}{8pt}\selectfont
\begin{alltt}
frp_api:event(\{\textcolor{magenta}{Network}, \textcolor{magenta}{entry}\}, get_new_data),
\end{alltt}
\end{block}

\begin{block}{Периодическое событие}
\fontsize{8pt}{8pt}\selectfont
\begin{alltt}
frp_api:timer(\{{\textcolor{magenta}{Network}, \textcolor{magenta}{entry}}\}, get_new_data, \textcolor{green}{15000}),
\end{alltt}
\end{block}
\end{frame}

\begin{frame}
\frametitle{Краткие выводы:}
\begin{itemize}
\item Использование модели акторов
\item Callback функции определяют правила преобразования данных 
\item События и их обработку задаёт пользователь
\item Изменение логики без изменения структуры
\end{itemize}
\end{frame}


\end{document}
